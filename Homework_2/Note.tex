\documentclass{article}

\usepackage[utf8]{inputenc}
\usepackage{CJKutf8}
\usepackage{amsmath}
\usepackage{amsfonts}
\usepackage{url}
%\usepackage{ffcode}
\usepackage{graphicx}

\begin{document}

\begin{CJK}{UTF8}{gbsn}

\section{2022年6月24日}

运行demo code,private score为0.46235。

\subsection{尝试不同的frames数目}

\begin{enumerate}
	\item 将\url{concat_nframes}设定为5,accuracy为0.567。
	\item 将\url{concat_nframes}设定为11,private score为0.63391。
	\item 将\url{concat_nframes}设定为21,会导致RAM爆掉。
\end{enumerate}

\subsection{尝试不同的hidden layers的数目}

\begin{enumerate}
	\item 将hidden layers从1改为2,private score为0.63567。没有提升。
	\item 加深到3层,将学习率从$10^{-4}$提高到$10^{-3}$,private score为0.66980,有明显提升。
	\item 继续加深到4层,Train Acc: 0.689845 Loss: 0.966410 | Val Acc: 0.667523 loss: 1.066279。现在已经开始overfitting了,之后使用别的技术来解决。
	\item 继续加深到5层,Train Acc: 0.688220 Loss: 0.976690 | Val Acc: 0.665157 loss: 1.069590。现在继续加深已经没有用了。
\end{enumerate}

\section{尝试batch normalisation}

首先将激活函数从ReLU改为sigmoid,结果明显变差。Train Acc: 0.620066 Loss: 1.236119 | Val Acc: 0.616463 loss: 1.251375。

仍然使用ReLU,在ReLU之前加入BatchNorm1d。Train Acc: 0.706134 Loss: 0.906614 | Val Acc: 0.680002 loss: 1.011591。结果变好。说明batch normalisation在ReLU上就有效果。Private score为0.68313。

尝试组合Sigmoid和BatchNorm1d。Train Acc: 0.674194 Loss: 1.026898 | Val Acc: 0.662758。结果是没有ReLU和BatchNorm1d好的。

\section{继续尝试不同的frames数目}

参考作业提示,当前应在\url{concat_nframes}这步尝试得不够。将RAM调大,然后继续5 layers + BatchNorm1D + ReLU。

\begin{enumerate}
	\item (control)将\url{concat_nframes}设定为11,之前的结果:Private score为0.68313。
	\item 将\url{concat_nframes}设定为21。Private score为0.71052。
	\item 将\url{concat_nframes}设定为41。Train Acc: 0.760101 Loss: 0.737564 | Val Acc: 0.709739 loss: 0.937353。Private score为0.71573。有进一步提高。
\end{enumerate}

\section{尝试加宽神经网络}

将\url{hidden_dim}从256加宽到1024,尝试能不能让training loss足够小。Train Acc: 0.937122 Loss: 0.174209 | Val Acc: 0.711997 loss: 1.452417。Private score:0.72765。进一步提高。

可以看到,这时候已经有明显的 overfitting 了。可以尝试一些 dropout。

\section{尝试dropout}

迭代10轮。

dropout$p = 0.5$。Train Acc: 0.665569 Loss: 1.090370 | Val Acc: 0.713079 loss: 0.913127。可以感觉是dropout加太强了。

dropout$p = 0.2$。Train Acc: 0.784093 Loss: 0.658822 | Val Acc: 0.750746 loss: 0.807976。Private score:0.75365。进一步提高。这里dropout起了作用。\textbf{这里的经验是dropout要起作用,需要神经网络表达能力,并且产生了明显的overfitting。dropout的目标是抑制overfitting,尽量让training loss和validation loss接近。}

\end{CJK}

\end{document}